\documentclass{article}[12pt]
\usepackage{graphicx}
\usepackage[thinlines]{easytable}
\usepackage[table]{xcolor}
\textheight=25.5cm \textwidth=18cm \topmargin=-1.8cm%
\evensidemargin=-0.8cm \oddsidemargin-0.8cm%

\begin{document}
\begin{center}
{\large Tiling with Polyominos}
\end{center}

Polyomino is a shape build from squares that touch on their sides. You are probably familiar with both dominoes and tetrominoes - have you ever played Tetris? We explore what can be built from polyominos - or what shapes can be tiled with polyomino-shaped tiles.

\vspace{5mm}

1. Can the following figure be made of 3x1 triominos?

\setlength{\unitlength}{12pt}
\begin{picture}(5, 5)
  \put(0, 0){\line(1, 0){4}}
  \put(0, 1){\line(1, 0){4}}
  \put(0, 2){\line(1, 0){4}}
  \put(0, 3){\line(1, 0){4}}
  \put(0, 4){\line(1, 0){4}}

  \put(0, 0){\line(0, 1){2}}
  \put(0, 3){\line(0, 1){1}}
  \put(1, 0){\line(0, 1){4}}
  \put(2, 0){\line(0, 1){4}}
  \put(3, 0){\line(0, 1){4}}
  \put(4, 0){\line(0, 1){4}}
\end{picture}

\vspace{3mm}

2. Can you tile a 3x5 rectangle with 1 3x1 triomino and 4 L-triominos? Can you tile it with 2 3x1 triominos and 3 L-triominos?

\vspace{3mm}

3. Arrange eight L-triominoes to create this shape (no tile on black square).

\setlength{\unitlength}{12pt}
\begin{picture}(10, 8)
  \put(6, 0){\line(1, 0){1}}
  \put(5, 1){\line(1, 0){3}}
  \put(4, 2){\line(1, 0){5}}
  \put(3, 3){\line(1, 0){7}}
  \put(3, 4){\line(1, 0){7}}
  \put(4, 5){\line(1, 0){5}}
  \put(5, 6){\line(1, 0){3}}
  \put(6, 7){\line(1, 0){1}}

  \put(3, 3){\line(0, 1){1}}
  \put(4, 2){\line(0, 1){3}}
  \put(5, 1){\line(0, 1){5}}
  \put(6, 0){\line(0, 1){7}}
  \put(7, 0){\line(0, 1){7}}
  \put(8, 1){\line(0, 1){5}}
  \put(9, 2){\line(0, 1){3}}
  \put(10, 3){\line(0, 1){1}}

  \put(6, 3){\line(1, 1){1}}
  \put(7, 3){\line(-1, 1){1}}
\end{picture}

\vspace{3mm}

4. Arrange one monomino, one L-triomino, and four straight triominoes to make a square.

\setlength{\unitlength}{12pt}
\begin{picture}(5, 5)
  \put(0, 0){\line(1, 0){4}}
  \put(0, 1){\line(1, 0){4}}
  \put(0, 2){\line(1, 0){4}}
  \put(0, 3){\line(1, 0){4}}
  \put(0, 4){\line(1, 0){4}}

  \put(0, 0){\line(0, 1){4}}
  \put(1, 0){\line(0, 1){4}}
  \put(2, 0){\line(0, 1){4}}
  \put(3, 0){\line(0, 1){4}}
  \put(4, 0){\line(0, 1){4}}
\end{picture}

\vspace{3mm}

5. Arrange the five tetrominoes (one of each) to create each of the shapes below.

\setlength{\unitlength}{12pt}
\begin{picture}(30, 8)
  \put(0, 0){\line(1, 0){3}}
  \put(0, 1){\line(1, 0){3}}
  \put(0, 2){\line(1, 0){3}}
  \put(0, 3){\line(1, 0){3}}
  \put(0, 4){\line(1, 0){3}}
  \put(0, 5){\line(1, 0){3}}
  \put(0, 6){\line(1, 0){3}}
  \put(0, 7){\line(1, 0){3}}

  \put(10, 0){\line(1, 0){3}}
  \put(10, 1){\line(1, 0){3}}
  \put(10, 2){\line(1, 0){3}}
  \put(10, 3){\line(1, 0){3}}
  \put(10, 4){\line(1, 0){3}}
  \put(10, 5){\line(1, 0){3}}
  \put(10, 6){\line(1, 0){3}}
  \put(10, 7){\line(1, 0){1}}
  \put(12, 7){\line(1, 0){1}}

  \put(20, 0){\line(1, 0){5}}
  \put(20, 1){\line(1, 0){5}}
  \put(20, 2){\line(1, 0){5}}
  \put(21, 3){\line(1, 0){4}}
  \put(22, 4){\line(1, 0){3}}
  \put(23, 5){\line(1, 0){2}}
  \put(24, 6){\line(1, 0){1}}

  \put(0, 0){\line(0, 1){7}}
  \put(1, 0){\line(0, 1){7}}
  \put(2, 0){\line(0, 1){7}}
  \put(3, 0){\line(0, 1){7}}

  \put(10, 0){\line(0, 1){7}}
  \put(11, 0){\line(0, 1){7}}
  \put(12, 0){\line(0, 1){7}}
  \put(13, 0){\line(0, 1){7}}

  \put(20, 0){\line(0, 1){2}}
  \put(21, 0){\line(0, 1){3}}
  \put(22, 0){\line(0, 1){4}}
  \put(23, 0){\line(0, 1){5}}
  \put(24, 0){\line(0, 1){6}}
  \put(25, 0){\line(0, 1){6}}
\end{picture}

\setlength{\unitlength}{12pt}
\begin{picture}(30, 8)
  \put(0, 0){\line(1, 0){3}}
  \put(0, 1){\line(1, 0){3}}
  \put(0, 2){\line(1, 0){3}}
  \put(0, 3){\line(1, 0){3}}
  \put(0, 4){\line(1, 0){3}}
  \put(0, 5){\line(1, 0){3}}
  \put(0, 6){\line(1, 0){3}}
  \put(0, 7){\line(1, 0){3}}

  \put(12, 0){\line(1, 0){2}}
  \put(11, 1){\line(1, 0){5}}
  \put(10, 2){\line(1, 0){7}}
  \put(10, 3){\line(1, 0){7}}
  \put(10, 4){\line(1, 0){6}}

  \put(20, 0){\line(1, 0){6}}
  \put(20, 1){\line(1, 0){6}}
  \put(20, 2){\line(1, 0){5}}
  \put(20, 3){\line(1, 0){5}}
  \put(20, 4){\line(1, 0){1}}
  \put(22, 4){\line(1, 0){3}}

  \put(0, 0){\line(0, 1){3}}
  \put(0, 4){\line(0, 1){3}}
  \put(1, 0){\line(0, 1){7}}
  \put(2, 0){\line(0, 1){7}}
  \put(3, 0){\line(0, 1){7}}

  \put(10, 2){\line(0, 1){2}}
  \put(11, 1){\line(0, 1){3}}
  \put(12, 0){\line(0, 1){4}}
  \put(13, 0){\line(0, 1){4}}
  \put(14, 0){\line(0, 1){4}}
  \put(15, 1){\line(0, 1){3}}
  \put(16, 1){\line(0, 1){3}}
  \put(17, 2){\line(0, 1){1}}

  \put(20, 0){\line(0, 1){4}}
  \put(21, 0){\line(0, 1){4}}
  \put(22, 0){\line(0, 1){4}}
  \put(23, 0){\line(0, 1){4}}
  \put(24, 0){\line(0, 1){4}}
  \put(25, 0){\line(0, 1){4}}
  \put(26, 0){\line(0, 1){1}}
\end{picture}

\vspace{3mm}

6. Which octominoes below can be made from dominoes?

\setlength{\unitlength}{12pt}
\begin{picture}(30, 6)
  \put(1, 0){\line(1, 0){2}}
  \put(0, 1){\line(1, 0){3}}
  \put(0, 2){\line(1, 0){3}}
  \put(0, 3){\line(1, 0){2}}
  \put(0, 4){\line(1, 0){1}}

  \put(7, 0){\line(1, 0){3}}
  \put(7, 1){\line(1, 0){3}}
  \put(7, 2){\line(1, 0){3}}
  \put(7, 3){\line(1, 0){2}}

  \put(16, 0){\line(1, 0){2}}
  \put(15, 1){\line(1, 0){3}}
  \put(14, 2){\line(1, 0){3}}
  \put(14, 3){\line(1, 0){3}}
  \put(14, 4){\line(1, 0){1}}

  \put(21, 0){\line(1, 0){4}}
  \put(21, 1){\line(1, 0){4}}
  \put(21, 2){\line(1, 0){3}}
  \put(22, 3){\line(1, 0){1}}

  \put(29, 0){\line(1, 0){1}}
  \put(28, 1){\line(1, 0){4}}
  \put(28, 2){\line(1, 0){4}}
  \put(29, 3){\line(1, 0){2}}
  \put(29, 4){\line(1, 0){1}}

  \put(0, 1){\line(0, 1){3}}
  \put(1, 0){\line(0, 1){4}}
  \put(2, 0){\line(0, 1){3}}
  \put(3, 0){\line(0, 1){2}}

  \put(7, 0){\line(0, 1){3}}
  \put(8, 0){\line(0, 1){3}}
  \put(9, 0){\line(0, 1){3}}
  \put(10, 0){\line(0, 1){2}}

  \put(14, 2){\line(0, 1){2}}
  \put(15, 1){\line(0, 1){3}}
  \put(16, 0){\line(0, 1){3}}
  \put(17, 0){\line(0, 1){3}}
  \put(18, 0){\line(0, 1){1}}

  \put(21, 0){\line(0, 1){2}}
  \put(22, 0){\line(0, 1){3}}
  \put(23, 0){\line(0, 1){3}}
  \put(24, 0){\line(0, 1){2}}
  \put(25, 0){\line(0, 1){1}}

  \put(28, 1){\line(0, 1){1}}
  \put(29, 0){\line(0, 1){4}}
  \put(30, 0){\line(0, 1){4}}
  \put(31, 1){\line(0, 1){2}}
  \put(32, 1){\line(0, 1){1}}
\end{picture}

\vspace{1cm}

7. Which of the shapes below can be made with one T-tetromino and four dominoes?

\setlength{\unitlength}{12pt}
\begin{picture}(32, 6)
  \put(0, 0){\line(1, 0){4}}
  \put(0, 1){\line(1, 0){4}}
  \put(0, 2){\line(1, 0){4}}
  \put(0, 3){\line(1, 0){4}}

  \put(7, 0){\line(1, 0){2}}
  \put(10, 0){\line(1, 0){1}}
  \put(7, 1){\line(1, 0){4}}
  \put(7, 2){\line(1, 0){4}}
  \put(7, 3){\line(1, 0){4}}
  \put(8, 4){\line(1, 0){1}}

  \put(14, 0){\line(1, 0){1}}
  \put(16, 0){\line(1, 0){2}}
  \put(14, 1){\line(1, 0){4}}
  \put(14, 2){\line(1, 0){4}}
  \put(14, 3){\line(1, 0){4}}
  \put(15, 4){\line(1, 0){1}}

  \put(21, 0){\line(1, 0){4}}
  \put(21, 1){\line(1, 0){4}}
  \put(21, 2){\line(1, 0){4}}
  \put(22, 3){\line(1, 0){3}}
  \put(22, 4){\line(1, 0){1}}

  \put(28, 0){\line(1, 0){4}}
  \put(28, 1){\line(1, 0){4}}
  \put(28, 2){\line(1, 0){4}}
  \put(29, 3){\line(1, 0){3}}
  \put(31, 4){\line(1, 0){1}}

  \put(0, 0){\line(0, 1){3}}
  \put(1, 0){\line(0, 1){3}}
  \put(2, 0){\line(0, 1){3}}
  \put(3, 0){\line(0, 1){3}}
  \put(4, 0){\line(0, 1){3}}

  \put(7, 0){\line(0, 1){3}}
  \put(8, 0){\line(0, 1){4}}
  \put(9, 0){\line(0, 1){4}}
  \put(10, 0){\line(0, 1){3}}
  \put(11, 0){\line(0, 1){3}}

  \put(14, 0){\line(0, 1){3}}
  \put(15, 0){\line(0, 1){4}}
  \put(16, 0){\line(0, 1){4}}
  \put(17, 0){\line(0, 1){3}}
  \put(18, 0){\line(0, 1){3}}

  \put(21, 0){\line(0, 1){2}}
  \put(22, 0){\line(0, 1){4}}
  \put(23, 0){\line(0, 1){4}}
  \put(24, 0){\line(0, 1){3}}
  \put(25, 0){\line(0, 1){3}}

  \put(28, 0){\line(0, 1){2}}
  \put(29, 0){\line(0, 1){3}}
  \put(30, 0){\line(0, 1){3}}
  \put(31, 0){\line(0, 1){4}}
  \put(32, 0){\line(0, 1){4}}
\end{picture}

\vspace{3mm}

9. Can you tile the figure below with dominoes?

\setlength{\unitlength}{12pt}
\begin{picture}(8, 8)
  \put(3, 0){\line(1, 0){3}}
  \put(2, 1){\line(1, 0){5}}
  \put(1, 2){\line(1, 0){6}}
  \put(1, 3){\line(1, 0){6}}
  \put(0, 4){\line(1, 0){7}}
  \put(0, 5){\line(1, 0){6}}
  \put(0, 6){\line(1, 0){4}}
  \put(0, 7){\line(1, 0){3}}

  \put(0, 4){\line(0, 1){3}}
  \put(1, 2){\line(0, 1){5}}
  \put(2, 1){\line(0, 1){6}}
  \put(3, 0){\line(0, 1){7}}
  \put(4, 0){\line(0, 1){6}}
  \put(5, 0){\line(0, 1){5}}
  \put(6, 0){\line(0, 1){5}}
  \put(7, 1){\line(0, 1){3}}

  \put(2, 3){\line(1, 1){1}}
  \put(3, 3){\line(-1, 1){1}}
\end{picture}

\end{document}
