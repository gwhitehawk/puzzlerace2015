\documentclass{article}[12pt]
\usepackage{graphicx}
\usepackage[thinlines]{easytable}
\usepackage[table]{xcolor}
\textheight=25.5cm \textwidth=18cm \topmargin=-1.8cm%
\evensidemargin=-0.8cm \oddsidemargin-0.8cm%

\begin{document}
\begin{center}
{\large Tiling and Parity Argument}
\end{center}

Coach-led activity:

Let us have an 8x8 square board like the one below (alternatively, 4x4). Is it possible to tile such a board with dominoes? Dominoes should cover the entire board without overlapping.

\setlength{\unitlength}{12pt}
\begin{picture}(9, 9)
  \put(0, 0){\line(1, 0){8}}
  \put(0, 1){\line(1, 0){8}}
  \put(0, 2){\line(1, 0){8}}
  \put(0, 3){\line(1, 0){8}}
  \put(0, 4){\line(1, 0){8}}
  \put(0, 5){\line(1, 0){8}}
  \put(0, 6){\line(1, 0){8}}
  \put(0, 7){\line(1, 0){8}}
  \put(0, 8){\line(1, 0){8}}

  \put(0, 0){\line(0, 1){8}}
  \put(1, 0){\line(0, 1){8}}
  \put(2, 0){\line(0, 1){8}}
  \put(3, 0){\line(0, 1){8}}
  \put(4, 0){\line(0, 1){8}}
  \put(5, 0){\line(0, 1){8}}
  \put(6, 0){\line(0, 1){8}}
  \put(7, 0){\line(0, 1){8}}
  \put(8, 0){\line(0, 1){8}}
\end{picture}

\vspace{3mm}

(Kids should work on the problem for a while but the answer is obviously yes.)

What if we cut one corner from the square, can it still be tiled with dominoes?

(Kids should come up with the solution quickly, it has odd number of unit squares, so no.)

What if we cut two opposite corner squares? Can it be tiled with dominoes?

(Kids work for longer, discuss ideas.)

If they intuitively know but can't formally show it, suggest coloring the board like a chessboard. Ask kids to pay attention to the properties that are true for each domino tile placed on the board.

(They should notice that each tile covers one black and one white square. Therefore, the resulting covered shape must have the number of both - the board with two corners cut doesn't. Therefore it is not possible to cover with dominoes.)

Present worksheet to practice understanding of the technique.
\end{document}
